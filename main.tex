\documentclass{article}
\usepackage{algorithm}
\usepackage{algpseudocode}
\usepackage{amsmath}
\usepackage{amssymb}

\begin{document}

\title{Complex Linear System Solvers}
\author{Your Name}
\date{\today}
\maketitle

\section{Introduction}
This document demonstrates linear solvers for complex linear system.

\section{Algorithms}
Below is an examples of an algorithm written in pseudocode.

\subsection{COCG and QMR\_SYM methods}

As we know, both COCG and QMR\_SYM are based on the well-lnown complex symmetisc Lanczos process, which is a special case for the Lanczos biorthogonalization process when the coefficient matrix is complex symmetric. This algorithm can be given as the following pseudocode.
\begin{algorithm}
\caption{The complex symmetric Lanczos process}
\begin{algorithmic}[1]
\label{Alg1}
\State Set $\beta_0=0$, ${\bf{v}}^L_{0} = \bf{0}$, $\bf{r}_0 \neq \bf{0} \in \mathbb{C}^{N}$. 
\State Set ${\bf{v}}^{L}_{1} = \bf{r}_0 / \left\| \bf{r}_0 \right\|$.
\For{$n=1, \, 2, \, \dots , \, m-1$}
\do{ :\\
${\bf{s}}^L_n = A {\bf{v}}^L_n$, \quad $\alpha_n = < {\bar{\bf{v}}}^L_n , {\bf{s}}^L_{n}>$. $^1$\\
${\Tilde{\bf{v}}}^L_{n+1} = {\bf{s}}^L_n - \alpha_n {\bf{v}}^L_n - \beta_{n-1} \bf{v}_{n-1}$. \\
$\beta_n = <{\bar{\Tilde{\bf{v}}}}^L_{n+1} , \, {\Tilde{\bf{v}}}^L_{n+1}>^{1/2}$.\\
${\bf{v}}^L_{n+1} = {\Tilde{\bf{v}}}^L_{n+1} / \beta_n$ .}
\EndFor
\end{algorithmic}
\end{algorithm}
If break down does not occur at step 7, the above algorithm generates a conjugate orthogonal basis of the Krylov subspace such that $<{\bar{\bf{v}}}^L_{i}, \, {\bf{v}}^L_{j}> = \delta_{ij}$. If we set $V^L = ({\bf{v}}^L_1 , \, \dots ,\, {\bf{v}}^L_n)$, similar to the Lanczos biorthogonalization process, Alg. \ref{Alg1} can also be written in matrix form. The matix form of Alg. \ref{Alg1} is given by $A V^L_n = V^L_{n+1} \Tilde{T}^L_{n} = V^L_n T^L_n + \beta_n {\bf{v}}^L_{n+1}{\bf{e}}^T_n$, where ${\bf{e}}_n = (0, \, 0,\, \dots ,\, 1)^T \in \mathbb{R}^n$, $(V^L_n )^T V^L_n = I_n$, $\Tilde{T}^L_n$, and $T^L_n$ are the $(n+1) \times n$ and $n\times n$ tridiagonal matrices whose diagonal entries are $\alpha_1 , \alpha_2, \dots $ and subdiagonal entries are $\beta_1 ,\, \beta_2 , \, \dots$ representing the recurrence coefficients of the complex symmetric Lanczos process. From the above, we see that $T^L_n$ is also complex symmetric. In the interest of counteraction against such breakdowns, refert oneself to remedies such as so-called look-ahead techniques (Van der Vorst, 2003; Saad, 2003), which can enhance stability while increasing cost modestly or others.  

\subsubsection{COCG Algorithm}
The COCG method is proposed as a useful Krylov subspace method for solving complex symmetric linear systems, and it is derived from the complex Lanczos process (see Alg \ref{Alg1}) of Krylov subspace. Here we describe below the COCG algorithm when applied to the system $A \bf{x} = \bf{b}$ with a complex symmetric matrix $A$. Its pseudocodes are given as follows.
\begin{algorithm}
\caption{The COCG method}
\begin{algorithmic}[1]
\label{Alg2}
\Require Give an initial guess ${\bf{x}}^G_0$ and compute ${\bf{r}}^G_0 = \bf{b}-A {\bf{x}}^{G}_0$. 
\State Set ${\bf{p}}^G_0 = {\bf{r}}^G_0, \, \rho_0 = <\bar{{\bf{r}}}^G_0 ,\, {\bf{r}}^G_0 >$ .
\For{$n=1, \, 2, \, \dots $ until convergence}
\do{ :\\
 ${\bf{q}}^G_{n-1}$.}
\EndFor
\end{algorithmic}
\end{algorithm}

\subsubsection{QMR\_SYM Algorithm}

\begin{algorithm}
\caption{The complex symmetric Lanczos process}
\begin{algorithmic}[3]
\Require $n \geq 0$ \Comment{Input condition}
\Ensure $y = x^n$ \Comment{Output condition}
\Function{Power}{$x,n$}
    \State $y \gets 1$
    \State $z \gets x$
    \State $m \gets n$
    \While{$m > 0$}
        \If{$m$ is even}
            \State $z \gets z \times z$
            \State $m \gets m / 2$
        \Else
            \State $y \gets y \times z$
            \State $m \gets m - 1$
        \EndIf
    \EndWhile
    \State \Return $y$
\EndFunction
\end{algorithmic}
\end{algorithm}

\end{document}